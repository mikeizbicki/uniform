\documentclass{article}

%%%%%%%%%%%%%%%%%%%%%%%%%%%%%%%%%%%%%%%%%%%%%%%%%%%%%%%%%%%%%%%%%%%%%%%%%%%%%%%%

\usepackage{geometry}
\usepackage{amsmath}
\usepackage{amssymb}
\usepackage{amsthm}
\usepackage{mathtools}
\usepackage{graphicx}
\usepackage{rotating}
\usepackage{xcolor}
\usepackage[inline]{enumitem}

%\usepackage{algorithm}
%\usepackage[noend]{algpseudocode}
%\usepackage{algorithmicx}
%\usepackage{algorithm2e}

%\usepackage[authordate,bibencoding=auto,strict,backend=biber,natbib]{biblatex-chicago}
\usepackage[round]{natbib}   % omit 'round' option if you prefer square brackets
\bibliographystyle{plainnat}

\usepackage{tikz}
\usepackage{pgfplots}
\pgfplotsset{width=7cm,compat=1.8}
\definecolor{darkgreen}{RGB}{0,127,0}

%%%%%%%%%%%%%%%%%%%%%%%%%%%%%%%%%%%%%%%%

\newtheorem{cor}{Corollary}
\newtheorem{theorem}{Theorem}
\newtheorem{lemma}{Lemma}

\theoremstyle{definition}
\newtheorem{definition}{Definition}
\newtheorem{example}{Example}
%\newcommand{\defn}[1]{\emph{#1}}
\newcommand{\defn}[1]{\textbf{#1}}

\DeclareMathOperator*{\argmin}{arg\,min}
\DeclareMathOperator*{\argmax}{arg\,max}
\DeclareMathOperator*{\vecspan}{span}
\DeclareMathOperator*{\affspan}{aff}
\DeclareMathOperator*{\subG}{subG}
\DeclareMathOperator*{\tr}{tr}
\DeclareMathOperator*{\E}{\mathbb{E}}
%\DeclareMathOperator*{\grad}{\nabla}
\newcommand{\grad}{\nabla\!}
\newcommand{\N}{\mathbb{N}}
\newcommand{\R}{\mathbb{R}}
\newcommand{\trans}[1]{{#1}^{\top}}

\newcommand{\ltwo}[1]{\lVert {#1} \rVert_2}
\newcommand{\set}{\mathcal}
\renewcommand{\vec}{\mathbf}

\newcommand{\W}{\mathcal{W}}
\newcommand{\X}{\mathcal{X}}
\newcommand{\Y}{\mathcal{Y}}
\newcommand{\Z}{Z}

\newcommand{\w}{W}
\newcommand{\what}{\hat\w}
\newcommand{\x}{\mathbf{x}}
\newcommand{\y}{y}

\newcommand{\loss}{\ell}
\newcommand{\reg}{r}

\newcommand{\+}{\oplus}
\newcommand{\meet}{\wedge}
\newcommand{\join}{\vee}
\newcommand{\dist}{d}
\newcommand{\breg}{D}

%\newcommand{\plots}[1]{}
\newcommand{\plots}[1]{#1}
\newcommand{\ignore}[1]{}
\newcommand{\fixme}[1]{\textbf{FIXME:} {#1}}

%%%%%%%%%%%%%%%%%%%%%%%%%%%%%%%%%%%%%%%%%%%%%%%%%%%%%%%%%%%%%%%%%%%%%%%%%%%%%%%%

\begin{filecontents}{paper.bib}

@article{harremoes2007information,
  title={Information topologies with applications},
  author={Harremo{\"e}s, Peter},
  journal={Entropy, Search, Complexity},
  volume={16},
  pages={113--150},
  publisher={Springer}
}

\end{filecontents}
\immediate\write18{bibtex paper}

%%%%%%%%%%%%%%%%%%%%%%%%%%%%%%%%%%%%%%%%%%%%%%%%%%%%%%%%%%%%%%%%%%%%%%%%%%%%%%%%

\title{Generalized Metric Spaces}
\author{Mike Izbicki}

\begin{document}

\maketitle

%%%%%%%%%%%%%%%%%%%%%%%%%%%%%%%%%%%%%%%%%%%%%%%%%%%%%%%%%%%%%%%%%%%%%%%%%%%%%%%%

\begin{definition}
    A \defn{lattice semigroup} is a set $L$ together with binary operations $\+,\meet,\join : L\times L\to L.$
\end{definition}

\begin{definition}
    A \defn{uniform space} is a set $X$ together with an associated lattice semigroup $L$ and a distance function $\dist : X\times X\to L$.
\end{definition}

\begin{example}
    Every metric space is a uniform space where the underlying lattice semigroup is the real number line $\R$.
\end{example}

\begin{example}
    Let $f : \R^d \to \R$ be a convex function.
    The \defn{Bregman divergence} $\breg_f : \R^d\times\R^d\to\R$ is defined to be
    \begin{equation}
        \breg_f(x,y) = f(x)-f(y) - \trans{(x-y)}\grad f(x)
        .
    \end{equation}
    In general, the Bregman divergence is not symmetric and does not satisfy the triangle inequality.
    We can, however, construct a uniform space for each Bregman divergence.

    Construct the Bregman lattice $L_f$ as follows.
\end{example}

\begin{example}
    Let $(X_1,\dist_1)$ and $(X_2,\dist_2)$ be two uniform spaces.
    Then the product uniform space is defined to be
\end{example}

%%%%%%%%%%%%%%%%%%%%%%%%%%%%%%%%%%%%%%%%

\section{References}

\citet{harremoes2007information} introduces the ``strong information topology'' induced by the KL divergence.
Do all divergences induce a topology?

%%%%%%%%%%%%%%%%%%%%%%%%%%%%%%%%%%%%%%%%%%%%%%%%%%%%%%%%%%%%%%%%%%%%%%%%%%%%%%%%

\clearpage
\bibliography{paper}

%%%%%%%%%%%%%%%%%%%%%%%%%%%%%%%%%%%%%%%%%%%%%%%%%%%%%%%%%%%%%%%%%%%%%%%%%%%%%%%%

\end{document}
