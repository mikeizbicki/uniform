\documentclass{article}

%%%%%%%%%%%%%%%%%%%%%%%%%%%%%%%%%%%%%%%%%%%%%%%%%%%%%%%%%%%%%%%%%%%%%%%%%%%%%%%%

\usepackage{geometry}
\usepackage{amsmath}
\usepackage{amssymb}
\usepackage{amsthm}
\usepackage{mathtools}
\usepackage{graphicx}
\usepackage{rotating}
\usepackage{xcolor}
\usepackage[inline]{enumitem}

%\usepackage{algorithm}
%\usepackage[noend]{algpseudocode}
%\usepackage{algorithmicx}
%\usepackage{algorithm2e}

%\usepackage[authordate,bibencoding=auto,strict,backend=biber,natbib]{biblatex-chicago}
\usepackage[round]{natbib}   % omit 'round' option if you prefer square brackets
\bibliographystyle{plainnat}

\usepackage{tikz}
\usepackage{pgfplots}
\pgfplotsset{width=7cm,compat=1.8}
\definecolor{darkgreen}{RGB}{0,127,0}

%%%%%%%%%%%%%%%%%%%%%%%%%%%%%%%%%%%%%%%%

\newtheorem{cor}{Corollary}
\newtheorem{theorem}{Theorem}
\newtheorem{lemma}{Lemma}

\theoremstyle{definition}
\newtheorem{remark}{Remark}
\newtheorem{definition}{Definition}
\newtheorem{example}{Example}
%\newcommand{\defn}[1]{\emph{#1}}
\newcommand{\defn}[1]{\textbf{#1}}

\DeclareMathOperator*{\argmin}{arg\,min}
\DeclareMathOperator*{\argmax}{arg\,max}
\DeclareMathOperator*{\vecspan}{span}
\DeclareMathOperator*{\affspan}{aff}
\DeclareMathOperator*{\subG}{subG}
\DeclareMathOperator*{\tr}{tr}
\DeclareMathOperator*{\E}{\mathbb{E}}
%\DeclareMathOperator*{\grad}{\nabla}
\newcommand{\grad}{\nabla\!}
\newcommand{\N}{\mathbb{N}}
\newcommand{\R}{\mathbb{R}}
\newcommand{\trans}[1]{{#1}^{\top}}

\newcommand{\ltwo}[1]{\lVert {#1} \rVert_2}
\newcommand{\set}{\mathcal}
\renewcommand{\vec}{\mathbf}

\newcommand{\W}{\mathcal{W}}
\newcommand{\X}{\mathcal{X}}
\newcommand{\Y}{\mathcal{Y}}
\newcommand{\Z}{Z}

\newcommand{\w}{W}
\newcommand{\what}{\hat\w}
\newcommand{\x}{\mathbf{x}}
\newcommand{\y}{y}

\newcommand{\loss}{\ell}
\newcommand{\reg}{r}

\newcommand{\+}{\oplus}
\newcommand{\meet}{\wedge}
\newcommand{\join}{\vee}
\newcommand{\dist}{d}
\newcommand{\breg}{D}

%\newcommand{\plots}[1]{}
\newcommand{\plots}[1]{#1}
\newcommand{\ignore}[1]{}
\newcommand{\fixme}[1]{\textbf{FIXME:} {#1}}

%%%%%%%%%%%%%%%%%%%%%%%%%%%%%%%%%%%%%%%%%%%%%%%%%%%%%%%%%%%%%%%%%%%%%%%%%%%%%%%%

\begin{filecontents}{paper.bib}

@article{george1994some,
  title={On some results in fuzzy metric spaces},
  author={George, A and Veeramani, P},
  journal={Fuzzy sets and systems},
  volume={64},
  number={3},
  pages={395--399},
  year={1994},
  publisher={Elsevier}
}

@article{harremoes2007information,
  title={Information topologies with applications},
  author={Harremo{\"e}s, Peter},
  journal={Entropy, Search, Complexity},
  volume={16},
  pages={113--150},
  publisher={Springer}
}

@article{huang2007cone,
  title={Cone metric spaces and fixed point theorems of contractive mappings},
  author={Huang, Long-Guang and Zhang, Xian},
  journal={Journal of mathematical Analysis and Applications},
  volume={332},
  number={2},
  pages={1468--1476},
  year={2007},
  publisher={Elsevier}
}

@article{azam2011common,
  title={Common fixed point theorems in complex valued metric spaces},
  author={Azam, Akbar and Fisher, Brian and Khan, M},
  journal={Numerical Functional Analysis and Optimization},
  volume={32},
  number={3},
  pages={243--253},
  year={2011},
  publisher={Taylor \& Francis}
}

@article{jankovic2011cone,
  title={On cone metric spaces: a survey},
  author={Jankovi{\'c}, Slobodanka and Kadelburg, Zoran and Radenovi{\'c}, Stojan},
  journal={Nonlinear Analysis: Theory, Methods \& Applications},
  volume={74},
  number={7},
  pages={2591--2601},
  year={2011},
  publisher={Elsevier}
}

@Article{pajoohesh2013kmetric,
author="Pajoohesh, H.",
title="k-metric spaces",
journal="Algebra universalis",
year="2013",
month="Feb",
day="01",
volume="69",
number="1",
pages="27--43",
abstract="In this paper, we give a new generalization of metric spaces called k-metric spaces. Our k-metrics are valued in lattice ordered groups, which allows us to talk about distance in non-abelian lattice ordered groups. We also discuss a class of (not necessarily abelian) lattice ordered groups in which every k-metric induces a topology. Then we show that every k-metric valued in the real numbers is metrizable. In the last section, we characterize intrinsic metrics on lattice ordered rings that are almost f-rings and prove that being an almost f-ring is necessary and sufficient for this characterization. Then we show that if a lattice ordered ring is representable, then every intrinsic metric is a k-metric.",
issn="1420-8911",
doi="10.1007/s00012-012-0218-8",
url="https://doi.org/10.1007/s00012-012-0218-8"
}

@article{shukla2014partial,
  title={Partial rectangular metric spaces and fixed point theorems},
  author={Shukla, Satish},
  journal={The Scientific World Journal},
  volume={2014},
  year={2014},
  publisher={Hindawi Publishing Corporation}
}

@article{paul2016extensions,
  title={Extensions of some fixed point results in fuzzy 2-metric spaces},
  author={Paul, Sudipta and Chetia, Sanchita},
  journal={International Journal of Engineering Science},
  volume={2917},
  year={2016}
}

\end{filecontents}
\immediate\write18{bibtex paper}

%%%%%%%%%%%%%%%%%%%%%%%%%%%%%%%%%%%%%%%%%%%%%%%%%%%%%%%%%%%%%%%%%%%%%%%%%%%%%%%%

\title{Generalized Metric Spaces}
\author{Mike Izbicki}

\begin{document}

\maketitle

%%%%%%%%%%%%%%%%%%%%%%%%%%%%%%%%%%%%%%%%%%%%%%%%%%%%%%%%%%%%%%%%%%%%%%%%%%%%%%%%

%%%%%%%%%%%%%%%%%%%%%%%%%%%%%%%%%%%%%%%%%%%%%%%%%%%%%%%%%%%%%%%%%%%%%%%%%%%%%%%%

\newpage

\begin{definition}
    A \defn{lattice semigroup} is a set $L$ together with binary operations $\+,\meet,\join : L\times L\to L.$
\end{definition}

\begin{definition}
    A \defn{uniform space} is a set $X$ together with an associated lattice semigroup $L$ and a distance function $\dist : X\times X\to L$.
\end{definition}

\begin{example}
    Every metric space is a uniform space where the underlying lattice semigroup is the real number line $\R$.
\end{example}

\begin{example}
    Let $f : \R^d \to \R$ be a convex function.
    The \defn{Bregman divergence} $\breg_f : \R^d\times\R^d\to\R$ is defined to be
    \begin{equation}
        \breg_f(x,y) = f(x)-f(y) - \trans{(x-y)}\grad f(x)
        .
    \end{equation}
    In general, the Bregman divergence is not symmetric and does not satisfy the triangle inequality.
    We can, however, construct a uniform space for each Bregman divergence.

    Construct the Bregman lattice $L_f$ as follows.
\end{example}

\begin{example}
    Let $(X_1,\dist_1)$ and $(X_2,\dist_2)$ be two uniform spaces.
    Then the product uniform space is defined to be
\end{example}

\begin{example}
    \citet{huang2007cone} introduces cone metric spaces.
    These seem to be the most general analogue of what I'm looking for.
    They restrict the lattice to sit within a Banach space.
    \citet{jankovic2011cone} provide a more updated survey of cone metrics.
    (It's really good!)
    Cone metrics were apparently previously called $K$-metrics.
    \citet{pajoohesh2013kmetric} contains some results on $k$-metrics, 
    but I'm not convinced they're entirely equivalent.

    \citet{azam2011common} introduce complex valued metric spaces and study fixed point theorems.
    \citet{paul2016extensions} studies 2-metrics in fuzzy metric spaces, and has lots of references.
    \citet{george1994some} more fuzzy metric spaces.
\end{example}

\begin{remark}
    Other generalizations of metric spaces include:
    partial metric spaces (where the self distance on a point need not be zero)
    and rectangular metric spaces (where the triangle inequality is 4-way) \citep{shukla2014partial}.
\end{remark}

%%%%%%%%%%%%%%%%%%%%%%%%%%%%%%%%%%%%%%%%

\section{References}

\citet{harremoes2007information} introduces the ``strong information topology'' induced by the KL divergence.
Do all divergences induce a topology?

%%%%%%%%%%%%%%%%%%%%%%%%%%%%%%%%%%%%%%%%%%%%%%%%%%%%%%%%%%%%%%%%%%%%%%%%%%%%%%%%

\clearpage
\bibliography{paper}

%%%%%%%%%%%%%%%%%%%%%%%%%%%%%%%%%%%%%%%%%%%%%%%%%%%%%%%%%%%%%%%%%%%%%%%%%%%%%%%%

\end{document}
